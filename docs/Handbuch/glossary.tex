% 
% funCKit - functional Circuit Kit
% Copyright (C) 2013  Lukas Elsner <open@mindrunner.de>
% Copyright (C) 2013  Peter Dahlberg <catdog2@tuxzone.org>
% Copyright (C) 2013  Julian Stier <mail@julian-stier.de>
% Copyright (C) 2013  Sebastian Vetter <mail@b4sti.eu>
% Copyright (C) 2013  Thomas Poxrucker <poxrucker_t@web.de>
% Copyright (C) 2013  Alexander Treml <alex.treml@directbox.com>
% 
% This program is free software: you can redistribute it and/or modify
% it under the terms of the GNU General Public License as published by
% the Free Software Foundation, either version 3 of the License, or
% (at your option) any later version.
% 
% This program is distributed in the hope that it will be useful,
% but WITHOUT ANY WARRANTY; without even the implied warranty of
% MERCHANTABILITY or FITNESS FOR A PARTICULAR PURPOSE.  See the
% GNU General Public License for more details.
% 
% You should have received a copy of the GNU General Public License
% along with this program.  If not, see <http://www.gnu.org/licenses/>.
% 


\newglossaryentry{cmd-esc}{
  name=ESC,
  description={
    Die ESC-Taste  }
  }

\newglossaryentry{cmd-scroll}{
  name=Scrollrad,
  description={
    Das Scrollrad der Maus. Bei Betätigung lässt es sich horizontal scrollen.
  }
}
\newglossaryentry{cmd-strg-scroll}{
  name=Strg+Scroll,
  description={
    Bei gedrückter Taste lässt sich bei Betätigung des Scrollrades zoomen.
   }
}
\newglossaryentry{cmd-alt-scroll}{
  name=Alt+Scroll,
  description={
    Bei gedrückter Taste lässt sich bei Betätigung des Scrollrades vertikal scrollen.
  }
}
\newglossaryentry{cmd-cmd}{
  name=Cmd,
  description={
Die Command-Taste bei Mac OS X. Wechselt in den Zoom-Modus und tritt in Verbindung mit anderen Tasten oft in anderen Kombinationen auf.
  }
}

\newglossaryentry{cmd-strg}{
  name=Strg,
  description={
Die Steuerung-Taste. Wechselt in den Zoom-Modus und tritt in Verbindung mit anderen Tasten oft in anderen Kombinationen auf.
  }
}

\newglossaryentry{cmd-strg-1}{
  name=Strg+1,
  description={
	Wechselt in den Selektions-Modus.
  }
}

\newglossaryentry{shift}{
  name=Shift (gedrückt),
  description={
	Wechselt temporär während die Taste gedrückt ist in den Selektions-Modus.
  }
}

\newglossaryentry{space}{
  name=Leertaste (gedrückt),
  description={
	Wechselt temporär während die Taste gedrückt ist in den Ansicht-verschieben-Modus.
  }
}

\newglossaryentry{newbricklist}{
  name=Neue-Baustein-Liste,
  description={
	Bietet eine Auswahl an Bausteinen die zum Erstellen ausgewählt werden können.
  }
}

\newglossaryentry{cmd-strg-2}{
  name=Strg+2,
  description={
	Wechselt in den Hinzufügen-Modus.
  }
}

\newglossaryentry{cmd-strg-3}{
  name=Strg+3,
  description={
	Wechselt zum Kabelwerkzeug.
  }
}

\newglossaryentry{cmd-strg-4}{
  name=Strg+4,
  description={
	Wechselt in den Ansicht verschieben - Modus.
  }
}
\newglossaryentry{cmd-strg-a}{
  name=Strg+A,
  description={
	Markiert alle Elemente im Gitternetz.
  }
}
\newglossaryentry{cmd-space}{
  name=Leertaste (gedrückt),
  description={
	Wechselt in den Ansicht verschieben - Modus.
  }
}


\newglossaryentry{cmd-strg-t}{
  name=Strg+T,
  description={
	Öffnet einen neuen Tab der aktuell geöffneten Schaltung.
  }
}

\newglossaryentry{cmd-strg-w}{
  name=Strg+W,
  description={
	Schließt den aktuellen Tab.
  }
}

\newglossaryentry{cmd-strg-n}{
  name=Strg+N,
  description={
	Neues Projekt öffnen.
  }
}

\newglossaryentry{cmd-strg-o}{
  name=Strg+O,
  description={
	Gespeichertes Projekt öffnen.
  }
}

\newglossaryentry{cmd-strg-s}{
  name=Strg+S,
  description={
	Projekt speichern.
  }
}

\newglossaryentry{cmd-strg-shift-s}{
  name=Strg+Umschalt+S,
  description={
	Projekt speichern unter.
  }
}

\newglossaryentry{cmd-strg-e}{
  name=Strg+E,
  description={
	Komponente exportieren.
  }
}

\newglossaryentry{cmd-strg-q}{
  name=Strg+Q,
  description={
	Programm beenden.
  }
}

\newglossaryentry{cmd-strg-z}{
  name=Strg+Z,
  description={
	Rückgängigmachen von Editieraktionen.
  }
}

\newglossaryentry{cmd-strg-y}{
  name=Strg+Y,
  description={
	Wiederherstellen von Editieraktionen.
  }
}

\newglossaryentry{cmd-strg-x}{
  name=Strg+X,
  description={
	Ausschneiden eines markiertes Bereichs.
  }
}

\newglossaryentry{cmd-strg-c}{
  name=Strg+C,
  description={
	Kopieren eines markiertes Bereichs.
  }
}

\newglossaryentry{cmd-strg-v}{
  name=Strg+V,
  description={
	Einfügen eines davor kopierten Bereichs.
  }
}


\newglossaryentry{cmd-entf}{
  name=Entf,
  description={
	Löschen eines markierten Bereichs.
  }
}
\newglossaryentry{cmd-f11}{
  name=F11,
  description={
	Anzeigen des Programms in Vollbild.
  }
}

\newglossaryentry{cmd-strg-plus}{
  name=Strg+Plus,
  description={
	Vergrößert die Ansicht des Editierpanels.
  }
}

\newglossaryentry{cmd-strg-minus}{
  name=Strg+Minus,
  description={
	Verkleinert die Ansicht des Editierpanels.
  }
}


\newglossaryentry{cmd-strg-0}{
  name=Strg+0,
  description={
	Zoom auf 100-Prozent.
  }
}

\newglossaryentry{cmd-strg-g}{
  name=Strg+G,
  description={
	Anzeige des Gitternetzes.
  }
}
\newglossaryentry{cmd-f1}{
  name=F1,
  description={
	Anzeige des Benutzerhandbuches.
  }
}

\newglossaryentry{Projekt}{
  name=Projekt,
  plural=Projekte,
  description={
       Ein Projekt besteht aus einer Hauptschaltung und bestimmten Zusatzinformationen wie einem Namen. Ein Projekt kann sowohl in eine Datei gespeichert werden als auch aus einer Datei geladen werden.
  }
}

\newglossaryentry{Komponente}{
  name=Komponente,
  description={
    Eine Komponente ist eine \gls{Schaltung} mit benannten Ein- und Ausgängen, die als Baustein in anderen Schaltungen verwendet werden kann.
  },
  plural=Komponenten
}
\newglossaryentry{Schaltung}{
  name=Schaltung,
  description={
    Eine Schaltung ist eine Menge von Bausteinen oder \glspl{Komponente}, die mit Leitungen verbunden sind.
  },
  plural=Schaltungen
}

\newglossaryentry{Tooltip}{
  name=Tooltip,
  description={
    Kästchen mit Informationen, die beim Überfahren eines Elements mit der Maus innerhalb der Benutzeroberfläche angezeigt werden.
  },
  plural=Tooltips
}
